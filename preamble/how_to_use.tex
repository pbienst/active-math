\chapter{How to use this text}

\newcommand{\iconoffset}{\raisebox{-.25\height}} 

Research has shown that the best way to learn and master a subject, and especially one like mathematics, is through \textbf{active learning}: you have to grapple with the material yourself, trying to wrap your head around it and trying to understand how it fits together with concepts you already know. Also, you have to try and apply your knowledge creatively in new situations, as opposed to just passively absorbing how someone else has done this before. Listening to a lecture, however expertly given, or reading through an explanation, however clearly written down, actually does surprisingly little to form the brain connections that are required for you to truly master the material.

This text is constructed based on these observations. First of all, what in other textbooks would be a long theoretical explanation, is split up using \textbf{cues}:

\begin{cue}
\textbf{Cues} will give you a couple of hints, and based on these, you should take the next steps in the derivation yourself. That way, you will recall the material much better, and you will also be able to identify the gaps in your knowledge.
\end{cue}

We strongly encourage you to really take the time to actually work through these steps yourself. The temptation to skip ahead to the result, which is presented just below, is obviously great, given the attractiveness of the path of least resistance. However, doing so would deprive you of a valuable opportunity to process the material yourself, and would only lead to a superficial illusion of competence, and ultimately, a waste of time.

A second ingredient for active learning are the diverse \textbf{exercises}:

\begin{exunmarked}
\textbf{Exercises} are crucial to allow you to practice what you have learned in many different circumstances. The majority of your effort should be spent on them. Also here, actually put in the effort yourself, as opposed to wasting your time by just looking up the solution without an honest attempt to wrestle with the problem.
\end{exunmarked}

The exercises are (somewhat subjectively) ranked in three different levels:

\begin{itemize}
\item \iconoffset\trivial : trivial problems, just to get you used to the new concepts.
\item \iconoffset\normal : problems with 'normal' difficulty, that you should be able to tackle after some thought.
\item \iconoffset\hard : hard problems, for when you like more of a challenge.
\end{itemize}

For some of these hard problems, hints\iconoffset\hint are provided, which lower the difficulty from a hard problem to a regular one. Feel free to use these hints, at least after banging your head against the problem for a while.

Many exercises also include a solution \iconoffset\solution, so that you can check your answers.

In the ebook version of this text, you can click on the\iconoffset\hint or \iconoffset\solution icons to view hints or solutions. Clicking on the exercise number in the solution or hint chapter will take you back to the exercise itself.

There is also an amount of \textbf{video material} to support this text (still growing). Clicking the video icon \iconoffset\youtube will take you to the video of the corresponding exercise or theory part. Exercise solution videos are only there as a safety net, in case you get stuck or when you want to double check your intermediate steps. The main purpose of the theory videos is to provide people who prefer listening to a human over reading text, with an alternative way of being exposed to the material. In case you understand everything just by working with the text, there is absolutely no need to waste your time watching the videos, since they do not contain extra material or new insights (unless explicitly mentioned). On the other hand, if your preferred way of working is through the videos, we still recommend you to read this text, as being able to concentrate on a non-trivial amount of textual material is a useful skill to cultivate.

\textbf{Review questions} are also included and refer to the main basic concepts of the material. Reviewing these from time to time is a helpful way to solidify the most important points in your mind.

We also encourage you to create your \textbf{own summaries} or mind maps of the discussed concepts (preferably in handwriting), as a way to further actively engage with the material and to help you see the connections between different parts of this text.

Finally, since this is so crucial, let us once more repeat the only way to master the material:
\\

\fbox{\textbf{Do not skip to solutions or explanations before putting in some serious effort!}}


%%% Local Variables:
%%% mode: latex
%%% TeX-master: "../main"
%%% End:
