\chapter{Preface}
\label{h:Preface}

The aim of this course is to expose students to various mathematical concepts often used in the field of photonics. Neither mathematical rigour nor extensive detail are envisaged, as the purpose of this course is to get students acquainted with a wide range of basic principles, which should allow them to independently refine their knowledge of mathematical topics that might emerge during their later career. As such, an important emphasis is placed on demonstrating the links of these mathematical concepts to concrete problems in the field of photonics.

The list of topics treated in this course also illustrates this focus on breath rather than depth:

The first four chapters deal with solving the wave equation in the case of time-harmonic optical signals, which gives rise to an equation in the complex plane. Chapter 1 therefore gives a brief introduction to complex calculus. Chapter 2 treats the solution of this equation for various special cases, for which we will need special functions and orthogonal polynomials. In order to solve this equation in the general case however, numerical techniques are required, so Chapter 3 will briefly discuss finite elements, finite differences and expansion in basis functions. Finally, Chapter 4 deals with the influence of periodicity and symmetry on the solutions of the complex wave equation.

After having treated time-harmonic systems, Chapter 5 will consider dynamical optical systems where the solutions can have a more general time dependence.

Finally, we wish to thank the people who provided feedback for this course and pointed out errors: Peter Vandersteegen, Kristof Vandoorne, Lieven Verslegers, Karel Van Acoleyen, Ziyue Zhang, Yancan Wu, Ruqi Shi, Martin Virte. We are also grateful to Wim Bogaerts for providing some of the figures for Chapter 4.\\[1cm]

\hspace{10cm} Peter Bienstman

%%% Local Variables:
%%% mode: latex
%%% TeX-master: "../main"
%%% End:
