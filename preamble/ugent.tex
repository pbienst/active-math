\chapter{Practical aspects related to the course}

The \textbf{contact hours} of this course are a mix between a flipped classroom and a study group. To explain the rational behind this, let's first look at some drawbacks of other models:

A traditional theory lecture is a quite inefficient way of transferring knowledge. Some people quip that a lecture is a way to transfer information from the notes of the instructor to the notes of the student, without passing through the brains of either...

In a flipped classroom model, students are asked to study the material beforehand, and then use the time in class to ask questions or solve exercises. This works fine, unless you had an exceptionally busy week (let's face it, life happens!) and could not preread the lecture notes. In that case, you would be wasting your time in class.

Solving exercises all together in class is also not an optimal use of everybody's time. If you're quick and you've finished all the assigned exercises, you're wasting your time at that point. If you need a bit more time to work on a exercise (which is perfectly fine!), when the instructor brings the solution to the board, they deny you an important learning opportunity, when you would have been able to solve it completely by yourself if only you had had a few more minutes.

This means that the optimal way to work through the material, which is most respectful of everybody's time, is essentially self-paced, instead of following a rhythm which is collectively imposed on the entire class. This self-pacing is facilitated through the existence of solutions, video clips, etc... (again, as a safety net to check your answers, not as a deceptive low-effort shortcut).

Therefore, in class, a combination of the following can happen:

\begin{itemize}
\item Anonymous multiple choice quizzes to help you assess your progress and to help the instructors identify problems.
\item Working in small groups through questions or exercises suggested by the instructor, if that's how you prefer to learn.
\item Working at your own pace through theory and exercises.   
\end{itemize}

An instructor is present to guide you and to answer any questions you might have.

A UGent icon \iconoffset\ugent\, indicates the minimum set of theory and exercises you should cover for this course. We do hope that you will go beyond that minimum, especially when it comes to the exercises.

To help you monitor your progress, the following table gives you a rough indication of what material you should have covered after each week:

\begin{center}
\begin{tabular}{ |c|c| } 
 \hline
  \textbf{Week} & \textbf{Material} \\
  \hline
 Week 1 & Section \ref{week1} \\ 
 Week 2 & Section \ref{week2} \\
 Week 3 & Section \ref{week3} \\
 Week 4 & Section \ref{week4} \\
 Week 5 & Section \ref{week5} \\
 Week 6 & Section \ref{week6} \\ 
 Week 7 & Section \ref{week7} \\
 Week 8 & Section \ref{week8} \\
 Week 9 & Section \ref{week9} \\
 Week 10 & Section \ref{week10}  \\
 Week 11 & Section \ref{week11} \\ 
 Week 12 & Section \ref{week12} \\
 \hline
\end{tabular}
\end{center}

Important: if you did your bachelor's degree at UGent, you have already seen the material on complex calculus. In this case, students should also tackle chapter \ref{h:kk-receiver} on the Kramers-Kronig receiver, which is a direct application of the concepts of complex calculus to optical telecommunications.

Your final grade for this course is a combination of the following three factors:

To gently encourage you to put in sustained effort throughout the semester, as opposed to wasting your time by cramming just before the exam, we ask that every week you email us the written traces of your engagement with this course (solved cues, exercises, ...). Do not spend any time cleaning these up, your draft notes are perfectly fine, even if they are difficulty to read or messy, as we only consider whether you were active, and not the details of your notes. So, this is very different from homework. Email either a scanned version of your paper notes, or pictures of the pages you took with your phone using a tool like Microsoft Lens or Google Lens. You are allowed to skip two weeks, to accommodate for exceptional circumstances. \textbf{Weekly mailing your notes} counts for 5 points out of 20 for the final grade.

We also ask that you provide \textbf{improvements to the course text}, either in the form of corrections, clearer formulations of things you found confusing, new exercises, new theory topics, ... . This counts for 2 points. The best entries will be incorporated in the next version of these notes.

Finally, the remainder of your grade consists of the exam, which is an \textbf{oral open book exam}. The main goal of the exam is to check that you thoroughly internalised the concepts in the course and that you can apply them. In terms of exercises, you are expected to be able to do trivial\iconoffset\trivial exercises on the fly. For problems belonging to the normal \iconoffset\normal\, category, which are a bit more involved, we are only interested in your general strategy to tackle them, and not the detailed calculations. The point of the exam is NOT to check your ability to accurately perform calculations under stress, nor to see how creative you can be under time pressure. 

%%% Local Variables:
%%% mode: latex
%%% TeX-master: "../main"
%%% End:
