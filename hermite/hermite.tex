\chapter{Hermite polynomials}
\label{h:hermite}

\begin{quote}
We are servants rather than masters in mathematics.

--- Charles Hermite
\end{quote}

\chaptertoc

In this chapter, we will look at Hermite polynomials, which are an example of a family of so-called orthogonal polynomials. They appear in the higher-order solutions of the paraxial wave equation.

After presenting the corresponding differential equation, we introduce their generating function and look at integrals involving these polynomials. We will also derive their orthogonality, as well as use them in series expansions.

\pagebreak

\sectionyoutubeugent{Paraxial wave equation}{z7KigqxDk1E}

Wave propagation is governed by the Helmholtz equation:

\begin{equation}
\nabla^2 \psi({\mathbf r}) + k^2 \psi({\mathbf r}) = 0
\end{equation}

The paraxial wave equation is an approximation to the Helmholtz equation when studying systems which have a well-defined optical axis, along which most of the light is propagating, like in a lens system.

Mathematically, we will express this by saying that the $z$-dependence of the field is mostly a plane wave $e^{-jkz}$ propagating along the $z$--direction. However, we will allow some other minor variations as a function of $z$, which should vary much slower compared to the plane-wave component:

\begin{equation}
\psi({\mathbf r}) = A({\mathbf r})e^{-jkz}
\label{eq-svea-ansatz}
\end{equation}

Since the envelope $A$ is varying slowly as a function of $z$, this is called the 'slowly varying envelope approximation' (SVEA).

\begin{cue}
Calculate $\partial^2 \psi / \partial z^2$.
\end{cue}

For the first derivative, we get

\begin{equation}
\frac{\partial \psi({\mathbf r})}{\partial z} = \frac {\partial A({\mathbf r})}{\partial z}e^{-jkz} -j k A({\mathbf r})e^{-jkz}
\end{equation} 

This means that

\begin{equation}
  \frac{\partial^2 \psi({\mathbf r})}{\partial z^2} = \frac{\partial^2 A({\mathbf r})}{\partial z^2}e^{-jkz} - 2 j k \frac{\partial A({\mathbf r})}{\partial z}e^{-jkz} - k^2 A({\mathbf r})e^{-jkz}
 \label{eq-svea-d2} 
\end{equation} 

\begin{cue}
If $A({\mathbf r})$ varies slowly as a function of $z$, what does that mean? Obviously, $\partial A / \partial z$ should be small, but small with respect to what? Make sure you compare against something with the correct dimensions.
\end{cue}

You might be tempted to write down that

\begin{equation}
  \left|\partial A({\mathbf r}) / \partial z\right| \ll 1
 \end{equation}

However, be aware that solutions to Maxwell's equations are only defined up to a scaling factor. So, multiplying $A$ by a big number would not fundamentally alter the solution from a physical point of view, but it would violate the inequality.

Next attempt:

\begin{equation}
 \left|\partial A({\mathbf r}) / \partial z\right| \ll \left|A({\mathbf r})\right|
\end{equation} 

At least that way, the criterion is unaffected by multiplication of $A$ by an arbitrary scaling factor. However, the dimensions are wrong: on the right-hand side, we need to divide by something which has a dimension of meter. Using the wavelength as a yard stick for this seems like a good idea. Since dividing by $\lambda$ is equivalent to multiplying by the wavevector $k$ (we don't worry about $2 \pi$ here), we get

\begin{equation}
\left|\partial A({\mathbf r}) / \partial z\right| \ll k \left|A({\mathbf r})\right| \label{eq-approx-paraxial-1}
\end{equation}

If we were to neglect $\partial A / \partial z$ completely, that would mean also neglecting the second-order derivative. That approximation is too coarse, since we would lose all $z$-dependence.

Therefore, let's try and take the derivative of Eq.~\ref{eq-approx-paraxial-1} with respect to $z$. Note that in general you cannot say that if $f(x) < g(x)$ then $f'(x) < g'(x)$. However, here $A$ is a slowly varying function, so we're assuming that taking derivatives of higher and higher order will result in smaller and smaller values. This means we can neglect $\partial^2 A / \partial z^2$ with respect to $j k \partial A / \partial z$ in Eq~\ref{eq-svea-d2}. 

\begin{cue}
Make this approximation and figure out what the Helmholtz equation reduces to.
\end{cue}

With this, and after factoring out the complex exponential, the Helmholtz equation becomes

\begin{equation}
\fbox{$\displaystyle
\nabla_T^2 A({\mathbf r}) -2jk \frac{\partial A({\mathbf r})}{\partial z} = 0
$}
\label{eq-paraxial}
\end{equation} 

Here, $\nabla_T^2$ stands for the transverse part $(\partial^2 / \partial x^2) + (\partial^2 / \partial y^2)$ of the Laplacian operator.

\pagebreak

It is possible to show that a solution to this equation is the following function, which describes a so-called Gaussian beam:

\begin{marginfigure}[-.7cm]
% credits: Wikipedia
% url: https://en.wikipedia.org/wiki/Carl_Friedrich_Gauss
\includegraphics{hermite/figures/c_gauss}
\caption{Carl Friedrich Gauss (1777-1855)}
\end{marginfigure}

\begin{equation}
A_G({\mathbf r}) = \frac{1}{q(z)}e^{-\frac{jk(x^2 + y^2)}{2q(z)}} \label{eq-gauss}
\end{equation}   

where $q(z)=z+jb_0$ and $b_0$ is a constant. 

In this context, the function $W(z)$ is defined as

\begin{equation}
W(z)=\sqrt{\frac{2 b_0}{k} \left(1 + \frac{z^2}{b_0^2}\right)} \label{eq-W}
\end{equation} 

$W(z)$ can be interpreted as the width of the Gaussian beam.

\pagebreak

\section{Hermite's differential equation}

\begin{marginfigure}[0.4cm]
  % credits: Wikipedia
  % url: https://en.wikipedia.org/wiki/Charles_Hermite
  \includegraphics{hermite/figures/c_hermite}
  \caption{Charles Hermite (1822–1901)}
  \end{marginfigure}

Let's look for higher-order solutions to the paraxial wave equation. 

Warning: the calculation ahead is extremely painful and does not really add anything interesting. The upcoming derivation is mainly given for completeness sake, and in the odd chance that someone really wants to know how everything fits together.

Let's propose a modulated version of the Gaussian beam as a trial solution:

\begin{equation}
A(x,y,z) = X\left({\frac{\sqrt{2}x}{W(z)}}\right) Y\left({\frac{\sqrt{2}y}{W(z)}}\right) e^{-jZ(z)} A_G(x,y,z) \label{eq-gauss-higher}
\end{equation} 

Let's also introduce $u = \sqrt{2} x / W(z)$ and  $v = \sqrt{2} y / W(z)$ as shortcuts. 

In Eq.~\ref{eq-gauss-higher}, $A_G$ is the Gaussian beam from Eq.~\ref{eq-gauss} and $X$, $Y$ and $Z$ are three real--valued functions that we still need to determine such that Eq.~\ref{eq-gauss-higher} satisfies the paraxial Helmholtz equation.


\begin{cue}
Calculate $\partial A^2 / \partial x^2$. Use the definition of the Gaussian beam Eq.~\ref{eq-gauss} to calculate $\partial A_G / \partial x$. 
\end{cue}
  
For the x--derivative of Eq.~\ref{eq-gauss-higher}, we get

\begin{equation}
\frac{\partial A}{\partial x} = \frac{\sqrt{2}}{W}X'Ye^{-jZ} A_G + XYe^{-jZ} \frac{\partial A_G}{\partial x} 
\end{equation} 

and

\begin{equation}
\frac{\partial^2 A}{\partial x^2} = \frac{2}{W^2}X''Ye^{-jZ} A_G  + 2\frac{\sqrt{2}}{W}X'Ye^{-jZ} \frac{\partial A_G}{\partial x}  + XYe^{-jZ} \frac{\partial^2 A_G}{\partial x^2}
\end{equation} 

Important: note that $X'$ is shorthand here for $X'(u)=dX/du$, where the prime denotes as usual the derivation with respect to the total argument, in this case $u=\sqrt{2}x/W(z)$.

Let's now use Eq.~\ref{eq-gauss} to calculate $\partial A_G / \partial x$:

\begin{equation}
\frac{\partial^2 A}{\partial x^2} = \frac{2}{W^2}X''Ye^{-jZ} A_G  - 2j k x \frac{\sqrt{2}}{qW}X'Ye^{-jZ}A_G  + XYe^{-jZ} \frac{\partial^2 A_G}{\partial x^2} \label{eq-hermite-gauss-1}
\end{equation} 

We can derive a similar equation for the $y$--derivative.

\begin{cue}
Calculate $\partial A / \partial z$.
\end{cue}

For the $z$--derivative, we get

\begin{align}
\frac{\partial A}{\partial z} =  -\frac{\sqrt{2}x W'}{W^2}X'Ye^{-jZ} A_G -\frac{\sqrt{2}y W'}{W^2}XY'e^{-jZ} A_G \nonumber \\ 
+ XY\left(-jZ'\right)e^{-jZ} A_G + XYe^{-jZ}\frac{\partial A_G}{\partial z} \label{eq-hermite-gauss-2}
\end{align} 

\begin{cue}
Plug these results into the paraxial wave equation.
\end{cue}

Let's substitute this in the paraxial equation Eq.~\ref{eq-paraxial}. Because $A_G$ is itself a solution of this equation, the last terms from Eq.~\ref{eq-hermite-gauss-1} and \ref{eq-hermite-gauss-2} cancel and we get:

\begin{align}
\frac{2}{W^2}\left(X''Y+XY''\right)e^{-jZ} A_G  \nonumber \\
-2jk \frac{\sqrt{2}}{Wq}\left(xX'Y+yXY'\right)e^{-jZ}A_G \nonumber \\
+2jk \frac{\sqrt{2} W'}{W^2}\left(xX'Y+yXY'\right)e^{-jZ}A_G \nonumber \\
-2jk XY\left(-jZ'\right)e^{-jZ} A_G = 0
\end{align}

\begin{cue}
Clean up, divide by $XY$, and then multiply by $W^2$.
\end{cue}

Getting rid of the common factors and dividing by $XY$, we get

\begin{equation}
\frac{1}{W^2}\left(\frac{X''}{X}+\frac{Y''}{Y}\right)  
- j k \left(\frac{\sqrt{2}}{Wq} - \frac{\sqrt{2}W'}{W^2}\right)\left(x\frac{X'}{X}+y\frac{Y'}{Y}\right -kZ' = 0
\end{equation} 

or

\begin{equation}
  \frac{1}{W^2}\left(\frac{X''}{X}+\frac{Y''}{Y}\right)  
  - j k \left(\frac{1}{q} - \frac{W'}{W^}\right)\frac{\sqrt{2}}{W}\left(x\frac{X'}{X}+y\frac{Y'}{Y}\right) -kZ' = 0
\end{equation} 

Multiplying by $W^2$, we get

\begin{equation}
\left(\frac{X''}{X}+\frac{Y''}{Y}\right)  
- j k  \left(\frac{W^2}{q} - WW'\right)\frac{\sqrt{2}}{W}\left(x\frac{X'}{X}+y\frac{Y'}{Y}\right)
-kW^2Z' = 0
\end{equation} 

\begin{cue}
Simplify $W^2/q - WW'$ based on the definition of $W$ and $q$. 
\end{cue}

From Eq.~\ref{eq-W}, we had

\begin{equation}
W(z)=\sqrt{\frac{2 b_0}{k} \left(1 + \frac{z^2}{b_0^2}\right)} 
\end{equation}

Also, $q(z)=z+jb_0$, so

\begin{equation}
\frac{W^2}{q} - WW' = \frac{\frac{2 b_0}{k} \left(1 + \frac{z^2}{b_0^2}\right)}{z+jb_0} - \sqrt{\frac{2 b_0}{k} \left(1 + \frac{z^2}{b_0^2}\right)} \cdot \frac{2\frac{b_0}{k} \cdot \frac{2z}{b_0^2}}{2\sqrt{\frac{2 b_0}{k} \left(1 + \frac{z^2}{b_0^2}\right)} }
\end{equation}

Cleaning up and getting rid of the complex denominator by multiplying with the complex conjugate, this becomes

\begin{equation}
\frac{\frac{2}{k b_0} \left(b_0^2 + z^2\right)}{z^2+b_0^2}(z-jb_0) - \frac{2z}{k b_0}
\end{equation}

Satisfyingly, this reduces to

\begin{equation}
\frac{2z}{k b_0}  + \frac{2}{k}(-j) - \frac{2z}{k b_0} = - \frac{2j}{k}
\end{equation}

Let's insert this result in the differential equation to simplify it. Let's also bring back the arguments of the different functions.

\begin{equation}
  \left(\frac{X''(u)}{X(u)}+\frac{Y''(v)}{Y(v)}\right)  
  - 2  \frac{\sqrt{2}}{W(z)}\left(x\frac{X'(u)}{X(u)}+y\frac{Y'(v)}{Y(v)}\right)
  -kW^2(z)Z'(z) = 0
  \end{equation} 

\begin{cue}
Use the definitions $u = \sqrt{2} x / W(z)$ and  $v = \sqrt{2} y / W(z)$. Then, perform separation of variables for the resulting differential equation. 
\end{cue}

Rearranging some terms, we get

\begin{equation}
\left[{\frac{X''(u)}{X(u)} - 2 u\frac{X'(u)}{X(u)}}\right] + 
\left[{\frac{Y''(v)}{Y(v)} - 2 v\frac{Y'(v)}{Y(v)}}\right] -kW^2(z)Z'(z) = 0
\end{equation} 

The left--hand side of this equation is a sum of three terms, each of which is a function of a single independent variable ($u$, $v$ and $z$ respectively). Therefore, each of these terms must be equal to a constant. Equating the first term to $-2\mu_1$ and the second to $-2\mu_2$, the third must be equal to $2(\mu_1+\mu_2)$. This separation of variables leads to the following ordinary differential equations:

\begin{equation}
X''(u) - 2 u X'(u) = - 2 \mu_1 X(u) 
\end{equation} 

\begin{equation}
Y''(v) - 2 v Y'(v) = - 2 \mu_2 Y(v)
\end{equation} 

\begin{equation}
-2b_0\left(1 + \frac{z^2}{b_0^2}\right)Z'(z) = 2(\mu_1+\mu_2)
\end{equation} 

From this, you can easily verify that 

$$Z(z) = -(\mu_1+\mu_2) \arctan\left(\frac{z}{b_0}\right)$$

because

\begin{equation}
Z'(z) = -(\mu_1+\mu_2) \frac{1}{1+\left(\frac{z}{b_0}\right)^2} \frac{1}{b_0}
\end{equation} 

The differential equations for $X$ and $Y$ are both of the following form: \marginnote{For aesthetical reasons we replaced $u$ by a generic argument $x$. Note however that this generic $x$ is different from the specific $x$ related to $u$ by the definition above.}

\begin{equation}
\fbox{$\displaystyle
X''(x) - 2 x X'(x) + 2 n X(x) = 0 \label{eq-diff-hermite-0}
$}
\end{equation} 

They don't have any obvious solutions at first sight. In the next sections, we will show that their solutions are \emph{Hermite polynomials}, and that $n$ is an integer.

In a similar vein to the treatment of Bessel functions, we will start by introducing a generating function and then continue to derive recurrence relations which will lead to a differential equation.

\pagebreak

\sectionyoutubeugent{Generating function for Hermite polynomials}{sN_rIAUIEhQ}

The generating function of the Hermite polynomials takes the following form:

\begin{equation}
g(x,t) = e^{-t^2 + 2tx} \label{eq-gen-hermite}
\end{equation}

The Hermite polynomials $H_n(x)$ are \emph{defined} from the the Laurent series in $t$ of $g(x,t)$ as 

\begin{equation}
\fbox{$\displaystyle
  g(x,t) = e^{-t^2 + 2tx} \stackrel{def}{=} \sum_{n = 0}^{\infty} H_n(x)\frac{t^n}{n!} \label{eq-g-hermite}
$}
\end{equation} 

\noindent\marginnote{An unfortunate side effect of the Latin alphabet having rather few letters...}Note the absence of a superscript in $H_n(x)$, which distinguishes them from the unrelated Hankel functions. The presence of $n!$ in the definition is purely cosmetic, and results in integer coefficients in the polynomial.

\begin{exer}
% difficulty: normal
% ugent
% youtube: lJ5NMKC-kcU   
\noindent\marginnote[0.7cm]{See the solution video for a graphical representation of the different terms in the product.}Show that
$$H_n(x) = \sum_{r=0}^{\lfloor n/2 \rfloor}(-1)^r {(2x)}^{n-2r} \frac{n!}{(n-2r)! r!}$$
\end{exer}

Here, $\lfloor x \rfloor$ is the largest integer smaller than or equal to $x$, i.e. $\lfloor 0.5 \rfloor=0$ and $\lfloor 1 \rfloor=1$.

With this formula, we can calculate the different polynomials (see Table~\ref{tab-hermite} and Fig.~\ref{fig-hermite}.)

\begin{figure}
\centering
\includegraphics[scale=0.7]{hermite/figures/hermite}
\caption{The Hermite polynomials $H_0(x)$, $H_1(x)$, $H_2(x)$.}
\label{fig-hermite}
\end{figure}

\begin{table}
\begin{align}
H_0(x) = & \, 1 \nonumber \\
H_1(x) = & \, 2x \nonumber \\
H_2(x) = & \, 4x^2-2 \nonumber \\
H_3(x) = & \, 8x^3-12x \nonumber \\
H_4(x) = & \, 16x^4-48x^2+12 \nonumber \\
H_5(x) = & \, 32x^5-160x^3+120x \nonumber \\
H_6(x) = & \, 64x^6-480x^4+720x^2-120 \nonumber
\end{align}
\caption{Hermite polynomials}
\label{tab-hermite}
\end{table}

\pagebreak

\begin{exer}
  % difficulty: normal
  % youtube: lu0xs5Aky00
Compared to the generating function of the Bessel functions, what feature of the generating function of the Hermite polynomials makes that Hermite polynomials are actually polynomials, i.e. that their series expansion terminates?
\end{exer}

\begin{exer}
  % difficulty: trivial
  % youtube: KiTa5xbfMd4
In Eq.~\ref{eq-g-hermite}, why are there no terms for negative powers of $t$?
\end{exer}

\begin{exer}
  % difficulty: hard
  % youtube: HS-UnSI-8ck 
Show that the coefficients of $H_n(x)$ are all integers.
\begin{hnt}
Factor out $n \choose r$.  
\end{hnt}
\end{exer}

\pagebreak


\sectionyoutubeugent{Recurrence relations for Hermite polynomials}{pajC-otGLNU}
\label{week5}

Similar to the treatment of Bessel functions, we can derive recurrence relations by differentiating the generating function.

\begin{cue}
By differentiating Eq.~\ref{eq-g-hermite} with respect to $t$ and derive a recurrence relation.
\end{cue}
 
By differentiating Eq.~\ref{eq-g-hermite} with respect to $t$, we get

\begin{equation}
(-2t+2x)e^{-t^2 + 2tx} = \sum_{n = 0}^{\infty} H_n(x) \frac{nt^{n-1}}{n!}
\end{equation} 

Substituting Eq.~\ref{eq-g-hermite} back in this, we get

\begin{equation}
(-2t+2x) \sum_{n = 0}^{\infty} H_n(x)\frac{t^n}{n!} = \sum_{n = 0}^{\infty} H_n(x) \frac{nt^{n-1}}{n!}
\end{equation} 

\noindent\marginnote{If you'd like more intermediate steps here, have a look at the video.}This leads to

\begin{equation}
-2  \frac{H_{n-1}(x)}{(n-1)!} + 2 x \frac{H_n(x)}{n!} = H_{n+1}(x) \frac{n+1}{(n+1)!}
\end{equation} 

where $n \geq 1$, or

\begin{equation}
H_{n+1}(x) = 2 x H_n(x) - 2 n H_{n-1}(x)
\label{eq-recur-hermite-1}
\end{equation} 
 

\begin{exer}
% difficulty: normal
% ugent
% youtube: sx23j_po8Zg  
Show that
$$\displaystyle H_n'(x) = 2nH_{n-1}(x)$$ \label{eq-recur-hermite-2}
\end{exer}


\pagebreak

\sectionyoutubeugent{Hermite's differential equation revisited}{tMVbpZUL2zE}

The next step is showing that the polynomials as defined through the generating function satisfy Hermite's differential equation. We will do this using the recurrence relations we derived from the generating function.

\begin{cue}
Take the derivative of Eq.~\ref{eq-recur-hermite-1} with respect to $x$ and use the recurrence relationship involving $H_n'(x)$ to calculate  $H_{n+1}'(x)$ and $H_n''(x)$. Then, recover Hermite's differential equation.
\end{cue}

Differentiating Eq.~\ref{eq-recur-hermite-1} with respect to $x$, we get

\begin{equation}
H_{n+1}'(x) = 2  H_n(x) + 2 x H_n'(x)- 2 n H_{n-1}'(x)
\end{equation} 

Using the results from Ex. \ref{eq-recur-hermite-2}, we have $H_{n+1}'(x) = 2(n+1)H_n(x)$ and $2 n H_{n-1}'(x) = H_n''(x)$:

\begin{equation}
2(n+1)H_n(x) = 2  H_n(x) + 2 x H_n'(x)- H_n''(x)
\end{equation} 

This reduces to

\begin{equation}
H_n''(x) - 2 x H_n'(x) + 2n H_n(x) = 0 \label{eq-diff-hermite-1}
\end{equation} 

which is indeed Eq.~\ref{eq-diff-hermite-0}.

This shows that the two definitions are equivalent.

\pagebreak

\sectionyoutubeugent{Gauss--Hermite modes}{LJoIkN1T7Ls}

Returning to our higher--order solutions of the paraxial wave equation, Fig.~\ref{fig-gauss-hermite} plots field profiles of some of these so--called \textbf{Gauss--Hermite modes}.

\begin{figure}
\centering
\includegraphics[width=5.5cm]{hermite/figures/gauss}
\caption{Cross-sections of some Gauss--Hermite modes in the $xy$ plane. Numbers $(n,m)$ above the modes correspond to the order of the polynomials in the $x$ resp. $y$-direction.}
\label{fig-gauss-hermite}
\end{figure}

\begin{cue}
What was the relationship again between the Hermite polynomials and the Gauss-Hermite modes?  
\end{cue}

Remember, the Gauss-Hermite modes are essentially a Gaussian mode multiplied by Hermite polynomials in the $x$ and $y$ directions. Therefore, they are characterised by two integer indices, indicating the order of the Hermite polynomials for each direction.

\begin{cue}
Why are the higher-order Gauss-Hermite modes broader?  
\end{cue}

The higher-order Gauss-Hermite modes are broader because they get multiplied by Hermite polynomials which 'emphasise' the values away from the origin.

\pagebreak


\sectionugent{Series expansion using Hermite polynomials}

Just as we did with Bessel functions, we can use Hermite polynomials to expand a function in a series. In order to do that, we will need to establish the correct orthogonality relation between Hermite polynomials and normalise them.

\subsectionyoutube{Orthogonality relationship for Hermite polynomials}{7OIAe4a4IU0}

Consider the following two differential equations:

\begin{equation}
\phi'' - 2 x \phi' + 2m \phi = 0 \label{eq-hermite-ortho-1}
\end{equation}
\begin{equation}
\psi'' - 2 x \psi' + 2n \psi = 0 \label{eq-hermite-ortho-2}
\end{equation}

The solutions of these are $H_m(x)$ and $H_n(x)$ respectively.

\begin{cue}
Multiply Eq.~\ref{eq-hermite-ortho-1} by $\psi e^{-x^2}$, Eq.~\ref{eq-hermite-ortho-2} by $\phi e^{-x^2}$ and subtract the two equations.
\end{cue}

Performing these manipulations, we get

\begin{equation}
e^{-x^2}\left(\phi''\psi -\psi''\phi\right)- e^{-x^2} 2 x \left(\phi'\psi -\psi'\phi\right)+ e^{-x^2}2(m-n)\phi\psi = 0
\end{equation} 

\begin{cue}
Identify a total differential in the first two terms of this equation.
\end{cue}

This can be written as

\begin{equation}
\left[e^{-x^2}\left(\phi'\psi -\psi'\phi\right)\right]' = e^{-x^2}2(n-m)\phi\psi
\end{equation} 

Integrating this between $-\infty$ and $\infty$ we get

\begin{equation}
2(n-m)\int_{-\infty}^{\infty}e^{-x^2}\phi\psi dx = \left[e^{-x^2}\left(\phi'\psi -\psi'\phi\right)\right]_{-\infty}^{+\infty}
\end{equation} 

The right hand side is equal to zero, because $e^{-{\infty}^2}$ goes to zero more quickly than any polynomial.

So in the end we get

\begin{equation}
\fbox{$\displaystyle
\int_{-\infty}^{\infty}e^{-x^2}H_n(x)H_m(x)dx = 0, \hspace{0.5cm} n \ne m \label{eq-hermite-ortho}
$}
\end{equation} 

This is the orthogonality relation for Hermite polynomials: they are orthogonal over the interval $[-\infty, \infty]$ with the weighting function $e^{-x^2}$.

\subsectionyoutube{Normalisation integral for Hermite polynomials}{4gLyl1FudXo}

To normalise the Hermite polynomials, we need to calculate

\begin{equation}
I = \int_{-\infty}^{\infty}e^{-x^2}H_n^2(x)dx
\end{equation}

\begin{cue}
Taking inspiration from the integrand, calculate $e^{-x^2}g(x,t)g(x,s)$. Use the definition of the generating function, integrate from $-\infty$ to $\infty$ and simplify. 
\end{cue}

Since we're taking the product of two series expansions, we obviously need to make the summation indices different:

\begin{align}
e^{-x^2} e^{-t^2 + 2tx} e^{-s^2 + 2sx} =& e^{-x^2} \left[\sum_{n = 0}^{\infty} H_n(x)\frac{t^n}{n!}\right]\left[\sum_{m = 0}^{\infty} H_m(x)\frac{s^m}{m!}\right] \nonumber \\
 =& \sum_{m, n = 0}^{\infty}e^{-x^2} H_n(x)\frac{t^n}{n!}H_m(x)\frac{s^m}{m!}
\end{align} 

When we integrate this over $x$ from $-\infty$ to $\infty$, the terms with $m \ne n$ on the right--hand side drop out because of the orthogonality relation Eq.~\ref{eq-hermite-ortho} ($t$ and $s$ are just constants as far as integration over $x$ is concerned):

\begin{equation}
\int_{-\infty}^{\infty} e^{-x^2} e^{-t^2 + 2tx} e^{-s^2 + 2sx} dx= \sum_{n = 0}^{\infty} \int_{-\infty}^{\infty} e^{-x^2} H_n^2(x)\frac{(st)^n}{n!n!} dx \label{eq-hermite-norm-1}
\end{equation}

Let's recall an interesting result:
\begin{exer}
  % difficulty: hard
  % youtube: 7ZOG3fnD584
Show that $\int_{-\infty}^{\infty}e^{-x^2}dx = \sqrt{\pi}$
\begin{hnt}
Write the integral as
$$\sqrt{\int_{-\infty}^{\infty} e^{-x^2}dx\int_{-\infty}^{\infty} e^{-y^2}dy}$$
and transform it to polar coordinates with $x^2+y^2=r^2$ and $dxdy = r dr d\theta$.
\end{hnt}
\end{exer}

\begin{cue}
Using the result from the exercise above, evaluate the left--hand side of Eq.~\ref{eq-hermite-norm-1}. (Try to get a factor $e^{-(x-s-t)^2}$ to appear in the integrand.)
\end{cue}

For the integral on the left--hand side of Eq.~\ref{eq-hermite-norm-1}, we get 

\begin{equation}
\int_{-\infty}^{\infty} e^{-x^2} e^{-t^2 + 2tx} e^{-s^2 + 2sx} dx 
  =  \int_{-\infty}^{\infty} e^{-(x-s-t)^2} e^{2st}dx = \sqrt{\pi} e^{2st}
  \label{eq-hermite-norm-2} 
\end{equation} 

\begin{cue}
Expand the right-hand side as a series. Compare this to the right-hand side of Eq.~\ref{eq-hermite-norm-1} to calculate the normalisation integral.
\end{cue}

We can write Eq.~\ref{eq-hermite-norm-2} as
\begin{equation}
  \int_{-\infty}^{\infty} e^{-x^2} e^{-t^2 + 2tx} e^{-s^2 + 2sx} dx 
    =  \sqrt{\pi} \sum_{n = 0}^{\infty} \frac{2^n{(st)}^n}{n!}  \label{eq-hermite-norm-3}
\end{equation} 
  
By equating like powers of $st$ in the right--hand sides of Eq.~\ref{eq-hermite-norm-1} and \ref{eq-hermite-norm-3}, we get the value of the normalisation integral:

\begin{equation}
\int_{-\infty}^{\infty} e^{-x^2} H_n^2(x) dx = 2^n n! \sqrt{\pi}
\end{equation} 

\subsectionyoutube{Series expansion using Hermite polynomials}{hlGd-1yzIe0}

With all the previous information, we can finally write the complete expression to expand a function $f(x)$ in a series of Hermite polynomials:

\begin{equation}
f(x) = \sum_{n=0}^{\infty}a_n H_n(x)
\end{equation} 

As before, taking the scalar product with an arbitrary but fixed basis function allows us to use the orthogonality relations to give us

\begin{equation}
\fbox{$\displaystyle
a_n = \frac{1}{2^n n!\sqrt{\pi}} \int_{-\infty}^{\infty} e^{-x^2} H_n(x) f(x) dx
$}
\end{equation} 


\pagebreak
\begin{exer}
% difficulty: trivial
% ugent
% youtube: 46Zklnt19wg 
  
Prove the following parity relation:
$$H_n(x) = (-1)^nH_n(-x)$$
a) by using the series expansion of Hermite polynomials

b) by replacing $t$ by $-t$ and $x$ by $-x$ in the generating function
\end{exer}


\begin{exer}
% difficulty: normal
% ugent
% youtube: SeHG8tfSXLc  
Use the definition of the generating function for Hermite polynomials to prove that
$$H_{2n}(0) = (-1)^n \frac{(2n)!}{n!}$$
$$H_{2n+1}(0) = 0$$

How does this property translate to the Gauss-Hermite modes?

\end{exer}

\begin{exer}
% difficulty: hard
% ugent
% youtube: C4sD9yhsueQ
Prove the Rodriguez formula for Hermite polynomials:
$$H_n(x) = (-1)^n e^{x^2}\frac{d^n}{d x^n}\left(e^{-x^2}\right)$$
\begin{hnt}
Verify this formula for a few values of $n$ to get a feeling for how it works, and then use mathematical induction.
\end{hnt}  
\end{exer}

\begin{exer}
  % difficulty: normal
  % youtube: _hZR2ao4RO4
For $0 \leq m \leq n-1$, use the Rodriguez formula to show that 
$$\int_{-\infty}^{\infty}x^m e^{-x^2} H_n(x) dx = 0$$
\end{exer}

\pagebreak

\begin{exer}
  % difficulty: normal
  % youtube: STRe4B6eZv0
Assume that based only on Hermite's differential equation, we managed to derive the recurrence relation involving $H_n'(x)$, as well as the values of $H_n(0)$. Based only on this information, derive the explicit form of the generating function $g(x,t)$ defined as

$$g(x,t) = \sum_{n = 0}^{\infty} H_n(x)\frac{t^n}{n!} $$

a) Take the derivative of $g(x,t)$ above with respect to $x$, use the recurrence relationship, and derive a first-order differential equation for $g(x,t)$.\\

b) Solve this equation to give

$$g(x,t) = e^{2tx} f(t)$$

c) Derive $f(t)$ by evaluating the expression for $x=0$.
\end{exer}

\begin{exer}
  % difficulty: hard
  % youtube: -Rykwhjj1JA
Prove that 
$$ | H_n(x) | \le  | H_n(jx) | $$
\begin{hnt}
  Remove factors from the series expansion of $ H_n(jx) $ so that it contains only positive terms.
\end{hnt}
\end{exer}


\begin{exer}
  % difficulty: hard
  % youtube: R5saFe_skGw 
Show that 
$$\left( 2x - \frac{d}{dx} \right)^n 1 = H_n(x)$$
\begin{hnt}
  Verify this formula for a few values of $n$ to get a feeling for how it works, and then use mathematical induction.
\end{hnt}
\end{exer}


\begin{exer}
  % difficulty: hard
  % youtube: g5kU_N383Dc
Calculate
$$ \int_{-\infty}^{\infty} e^{-x^2} x^2 H_n^2(x) dx $$
This integral occurs in the calculation of the mean-square displacement of a quantum oscillator.
\begin{sol}
$$ = 2^{n-1} (2n + 1) n! \sqrt{\pi}$$
\end{sol}
\begin{hnt}
Expand $x H_n(x)$ using a recurrence relation and apply the orthogonality conditions.
\end{hnt}  
\end{exer}

\pagebreak

\begin{exer}
  % difficulty: hard
  % youtube: LxtS5QezdyQ 
Expand the following integral in a power series in $t$

$$ \int_{-\infty}^{\infty} e^{-x^2/2} g(x,t) dx$$

and use the result to calculate

$$\int_{-\infty}^{\infty} e^{-x^2/2} H_{m}(x) dx $$

\begin{sol}
$$\int_{-\infty}^{\infty} e^{-x^2/2} H_{2m}(x) dx = \frac{(2m)!}{m!} \sqrt{2\pi} $$
$$\int_{-\infty}^{\infty} e^{-x^2/2} H_{2m+1}(x) dx = 0 $$
\end{sol}
\end{exer}

\begin{marginfigure}[5.0cm]
  % credits: Wikipedia
  % url: https://en.wikipedia.org/wiki/Olinde_Rodrigues
  \includegraphics{hermite/figures/o_rodrigues}
  \caption{Olinde Rodrigues (1795–1851)}
  \end{marginfigure}

\begin{exer}
  % difficulty: hard
  % youtube: DP4aVSd3YBY
Multiply the generating function by $t^{-k-1}$ and integrate over the unit circle in the complex $t$-plane to show that

$$H_k(x) =\frac{k!}{2 \pi j} \oint \frac{e^{-t^2 +2tx}}{t^{k+1}}dt $$

Then, show by direct substitution that this expression satisfies Hermite's differential equation.

\begin{hnt}
  Try to write the integrand as a total differential of an exponential divided by $t^k$.
\end{hnt}
    
\end{exer}


\section*{Review questions}

\begin{itemize}
\item What are Hermite polynomials?  
\item How can a generating function be used to define Hermite polynomials?
\item What is the relationship between Hermite polynomials and Gauss-Hermite modes?  
\item Under what scalar product are Hermite polynomials orthogonal?
\item For which physical systems would you use a series expansion based on Hermite polynomials?  
\end{itemize}



%%% Local Variables:
%%% mode: latex
%%% TeX-master: "../main"
%%% End: